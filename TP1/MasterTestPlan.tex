\documentclass[]{scrreprt}
\usepackage[spanish]{babel}
\usepackage[utf8]{inputenc}
\usepackage{listings}
\usepackage{underscore}
\usepackage{multirow}
\usepackage[table,xcdraw]{xcolor}
\usepackage{adjustbox}
\usepackage[bookmarks=true]{hyperref}
\hypersetup{
    bookmarks=false,    % show bookmarks bar?
    pdftitle={Master Test Plan},    % title
    pdfauthor={Ivan Szkrabko},                     % author
    pdfsubject={TeX and LaTeX},                        % subject of the document
    pdfkeywords={TeX, LaTeX}, % list of keywords
    colorlinks=true,       % false: boxed links; true: colored links
    linkcolor=blue,       % color of internal links
    citecolor=black,       % color of links to bibliography
    filecolor=black,        % color of file links
    urlcolor=purple,        % color of external links
    linktoc=page            % only page is linked
}
\def\myversion{2.0 }
\title{
\flushleft
\Huge{Master test plan}\\
\vspace{1cm}
para\\
\vspace{1cm}
Realidad aumentada para la optimización de procedimientos batch en la industria\\
\vspace{1cm}
\LARGE{Versión 1.0\\}
\vspace{1cm}
Realizado por Iván Szkrabko\\
}
\begin{document}
\maketitle
\tableofcontents
\chapter{Características de calidad}
Se enumeran las características de calidad más relevantes para el proyecto:

\begin{itemize}
	\item Funcionalidad
	\\ El objetivo del proyecto es hacer una aplicación funcional para la operación de la planta.
	\item Usabilidad
	\\ La aplicación debe ser fácil de utilizar e intuitiva, esto es parte de los requerimientos.
	\item Confiabilidad
	\\ Es una característica importante para asegurar la validez de los datos que el operador es capaz de 				interpretar de los distintos elementos de la planta.
\end{itemize}
\chapter{Determinación de la importancia relativa}
Se asignan las siguientes importancias a las características de calidad:
\begin{itemize}
	\item Funcionalidad: 40\%
	\item Usabilidad:    30\%
	\item Confiabilidad: 30\%
\end{itemize}
\chapter{Características de calidad según niveles de prueba}
\begin{table}[h]
\begin{tabular}{|l|c|c|c|}
\hline
& \multicolumn{1}{l|}{Funcionalidad} & \multicolumn{1}{l|}{Usabilidad} & \multicolumn{1}{l|}{Confiabilidad} \\ \hline
Importancia Relativa (\%) & 40                                 & 30                              & 30                                 \\ \hline
Test unitario             & ++                                 &   *                             & * \\ \hline
Test de integracion de SW &   *                                &  *                              & ++                                 \\ \hline
Test de sistema           & +                                  & ++                              & * \\ \hline
\end{tabular}
\end{table}

Referencias:
\begin{table}[h]
\begin{tabular}{rl}
\textbf{++} & :La característica de calidad es muy importante. \\
\textbf{+}  & :La característica de calidad es relevante.      \\
\textbf{*}  & :La característica de calidad es irrelevante.     
\end{tabular}
\end{table}

\begin{itemize}
	\item Funcionalidad/Test unitario (++)
	\\ Los test unitarios se encargaran de validar las lógicas de navegación de la interfaz de usuario.
	\item Confiabilidad/Test de integración de software (++)
	\\ La comunicación de las distintas interfaces de software debe ser testeada para asegurar su confiabilidad
	\item Usabilidad/Test de sistema (++)
	\\ El sistema en su conjunto debe ser fácil de utilizar y consistente, se analizaran los tiempos de respuesta 		de las interfaces principalmente.
	\item Funcionalidad/Test de sistema (+)
	\\ El sistema en su conjunto debe ser funcional. 						
\end{itemize}


\chapter{División del sistema en subsistemas}
El sistema se dividirá en subsistemas que pueden probarse por separado, estas subdivisiones coinciden con la arquitectura de la solución planteada. Cada subsistema cumple una funcionalidad especifica y debe garantizarse su funcionamiento así como las características de calidad establecidas. A continuación se muestra la subdivisión del sistema:
\begin{itemize}
	\item Interfaz usuario
	\\Presentación de la interfaz visual al operador.
	\item Servidor Local
	\\Procesar input del usuario, generar comandos para el sistema de control.
	\item Cliente OPC
	\\Envía los comandos al sistema de control y recibe información del mismo.
	\item Sistema control
	\\Ejecuta los comandos validos y reporta las señales de los distintos elementos de control.
\end{itemize}

\chapter{Importancia relativa de los subsistemas}
\begin{table}[h]
\centering
\begin{tabular}{|c|c|}
\hline
\textbf{Subsistema}          & \multicolumn{1}{l|}{\textbf{Importancia Relativa(\%)}} \\ \hline
\textbf{Interfaz de Usuario} & 20                                                 \\ \hline
\textbf{Servidor Local}      & 20                                                 \\ \hline
\textbf{Cliente OPC}         & 40                                                 \\ \hline
\textbf{Sistema de control}  & 20                                                 \\ \hline
\end{tabular}
\end{table}

\chapter{Importancia de test por combinaciones de subsistema}
\begin{table}[h]
\begin{adjustbox}{width=1.2\textwidth}
\begin{tabular}{|c|c|c|c|c|c|}
\hline
\textbf{Subsistema}& \multicolumn{1}{l|}{\textbf{Importancia Relativa (\%)}} & \multicolumn{1}{l|}{\textbf{Interfaz de Usuario}} & \multicolumn{1}{l|}{\textbf{Servidor Local}} & \multicolumn{1}{l|}{\textbf{Cliente OPC}} & \multicolumn{1}{l|}{\textbf{Sistema de control}} \\ \hline
\textbf{Importancia Relativa (\%)} & 100                                                     & 20                                                & 20                                           & 40                                        & 20                                               \\ \hline
\textbf{Funcionalidad}             & 20                                                      & +                                                 & ++                                           & +                                         & ++                                               \\ \hline
\textbf{Usabilidad}                & 20                                                      & ++                                                & *                                            & *                                         & *                                                \\ \hline
\textbf{Confiabilidad}             & 40                                                      & +                                                 & +                                            & ++                                        & +                                                \\ \hline
\end{tabular}
\end{adjustbox}
\end{table}

Referencias:
\begin{table}[h]
\begin{tabular}{rl}
\textbf{++} & :La característica de calidad es muy importante. \\
\textbf{+}  & :La característica de calidad es relevante.      \\
\textbf{*}  & :La característica de calidad es irrelevante.     
\end{tabular}
\end{table}

\chapter{Tecnicas de test aplicadas}

\begin{table}[h]
\begin{tabular}{|c|c|c|c|c|}
\hline
\textbf{Subsistema}                 & \multicolumn{1}{l|}{\textbf{Interfaz de Usuario}} & \multicolumn{1}{l|}{\textbf{Servidor Local}} & \multicolumn{1}{l|}{\textbf{Cliente OPC}} & \multicolumn{1}{l|}{\textbf{Sistema de control}} \\ \hline
\textbf{SST}   & +                                                 &                                              &                                           &                                                  \\ \hline
\textbf{CTM} &                                                   & +                                            &                                           & +                                                \\ \hline
\textbf{ECT} &                                                   &                                              & +                                         &                                                  \\ \hline
\end{tabular}
\end{table}

\begin{table}[h]
\begin{tabular}{rl}
\textbf{SST} & :State transition testing. \\
\textbf{CTM}  & :Classification-tree method.      \\
\textbf{ECT}  & :Elementary comparison test.     
\end{tabular}
\end{table}

\end{document}